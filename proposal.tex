%%%%%%%%%%%%%%%%%%%%%%%%%%%%%%%%%%%%%%%%%%%%%%%%%%%%%%%%%%%%%%%
%%%  notes
%%%%%%%%%%%%%%%%%%%%%%%%%%%%%%%%%%%%%%%%%%%%%%%%%%%%%%%%%%%%%%%

\documentclass[onecolumn,fleqn,notitlepage,secnumarabic]{revtex4}

% special 
\usepackage{ifthen}
\usepackage{ifpdf}
\usepackage{float}
\usepackage{color}

% fonts
\usepackage{latexsym}
\usepackage{amsmath} 
\usepackage{amssymb} 
\usepackage{bm}
\usepackage{wasysym}


\ifpdf
\usepackage{graphicx}
\usepackage{epstopdf}
\else
\usepackage{graphicx}
\usepackage{epsfig}
\fi

% packages added by jarondl

\usepackage{verbatim} % for multiline comments
\usepackage{natbib} % change the bibliography style 
\usepackage{fancybox} % allows putting boxes with borders
\usepackage{cmap}  % for making pdf mathmode searchable
\usepackage[pdftitle={proposal}]{hyperref}  % for hyperlinks in biblio. should be called last?

%%%%%%%%%%%%%%%%%%%%%%%%%%%%%%%%%%%%%%%%%%%%%%%%%%%%%%%%%%%%%%%%


% math symbols I
\newcommand{\sinc}{\mbox{sinc}}
\newcommand{\const}{\mbox{const}}
\newcommand{\trc}{\mbox{trace}}
\newcommand{\intt}{\int\!\!\!\!\int }
\newcommand{\ointt}{\int\!\!\!\!\int\!\!\!\!\!\circ\ }
\newcommand{\ar}{\mathsf r}
\newcommand{\im}{\mbox{Im}}
\newcommand{\re}{\mbox{Re}}

% math symbols II
\newcommand{\eexp}{\mbox{e}^}
\newcommand{\bra}{\left\langle}
\newcommand{\ket}{\right\rangle}

% Mass symbol
\newcommand{\mass}{\mathsf{m}} 
\newcommand{\rdisc}{\epsilon} 

% more math commands
\newcommand{\tbox}[1]{\mbox{\tiny #1}}
\newcommand{\bmsf}[1]{\bm{\mathsf{#1}}} 
\newcommand{\amatrix}[1]{\begin{matrix} #1 \end{matrix}} 
\newcommand{\pd}[2]{\frac{\partial #1}{\partial #2}}

% equations
\newcommand{\be}[1]{\begin{eqnarray}\ifthenelse{#1=-1}{\nonumber}{\ifthenelse{#1=0}{}{\label{e#1}}}}
\newcommand{\ee}{\end{eqnarray}} 
\newcommand{\beq}{\be{1}}
\newcommand{\eeq}{\ee} 

% arrangement
\newcommand{\hide}[1]{}
\newcommand{\drawline}{\begin{picture}(500,1)\line(1,0){500}\end{picture}}
\newcommand{\bitem}{$\bullet$ \ \ \ }
\newcommand{\Cn}[1]{\begin{center} #1 \end{center}}
\newcommand{\mpg}[2][1.0\hsize]{\begin{minipage}[b]{#1}{#2}\end{minipage}}
\newcommand{\mpgt}[2][1.0\hsize]{\begin{minipage}[t]{#1}{#2}\end{minipage}}
\newcommand{\sect}[1]{{\bf #1.-- }}



%fminipage using fancybox package
\newenvironment{fminipage}%
  {\begin{Sbox}\begin{minipage}}%
  {\end{minipage}\end{Sbox}\fbox{\TheSbox}}


%%%%%%%%%%%%%%%%%%%%%%%%%%%%%%%%%%%%%%%%%%%%%%%%%%%%%%%%%%%%%%%%%%%%%%%%%%%

% Page setup
\setlength{\parindent}{0cm} 
\setlength{\parskip}{0.3cm} 

%%% Sections. The original revtex goes like this:
%\def\section{%
%  \@startsection
%    {section}%
%    {1}%
%    {\z@}%
%    {0.8cm \@plus1ex \@minus .2ex}%
%    {0.5cm}%
%    {\normalfont\small\bfseries}%
%}%
%\def{ \bf %
%  \@startsection
%    {subsection}%
%    {2}%
%    {\z@}%
%    {.8cm \@plus1ex \@minus .2ex}%
%    {.5cm}%
%    {\normalfont\small\bfseries}%
%}%
%%%%%%% And our version goes like this:
\makeatletter
\def\section{%
  \@startsection
    {section}%
    {1}%
    {\z@}%
    {0.8cm \@plus1ex \@minus .2ex}%
    {0.5cm}%
    {\Large\bf }%
}%
\def\subsection{%
  \@startsection
    {subsection}%
    {2}%
    {\z@}%
    {.8cm \@plus1ex \@minus .2ex}%
    {.5cm}%
    {\normalfont\small\bfseries}%
}%
%%%%%%%%%%  Here we deal with capitalization. The original revtex first, and then our version.
%\def\@hangfrom@section#1#2#3{\@hangfrom{#1#2}\MakeTextUppercase{#3}}%
%\def\@hangfroms@section#1#2{#1\MakeTextUppercase{#2}}%
\def\@hangfrom@section#1#2#3{\@hangfrom{#1#2}{#3}}%
\def\@hangfroms@section#1#2{#1{#2}}%
\makeatother


%%%%%%%%%%%%%%%%%%%%%%%%%%%%%%%%%%%%%%%%%%%%%%%%%%%%%%%%%%%%%
%%%%%%%%%%%%%%%%%%%%%%%%%%%%%%%%%%%%%%%%%%%%%%%%%%%%%%%%%%%%%
\begin{document}

\title{Heat transport in low-dimensional random harmonic networks}


\author{ Isaac Weinberg\\Adviser: Doron Cohen }
\affiliation{Physics Department, Ben Gurion University of the Negev}
\date{\today}
\maketitle

%%%%%%%%%%%%%%%%%%%%%%%%%%%%%%%%%%%%%%%%%%%%%%%%%%%%%%%%%%%%%%%%%%%%%%%%%%%%%%%%%%%%%%%%%%%%
%%%%%%%%%%%%%%%%%%%%%%%%%%%%%%%%%%%%%%%%%%%%%%%%%%%%%%%%%%%%%%%%%%%%%%%%%%%%%%%%%%%%%%%%%%%
%%%%%%%%%%%%%%%%%%%%%%
\section{Background}

The theory of phononic heat conduction in disordered low-dimensional networks is a central theme of research in recent years \cite{LLP03,
D08,LRWZHL12}. The interest in this theme is not only purely academic, but it is also motivated by the ongoing developments in nanotechnology.
In spite of the recent research efforts, the understanding of thermal transport is still at its infancy. This becomes more obvious if one compares 
with the achievements that have been experienced during the last fifty years in understanding and managing electron transport. In this respect 
even the microscopic laws that govern heat conduction in low dimensional systems have only recently start being scrutinized via both theoretical, 
numerical and experimental studies \cite{LLP03,D08,COGMZ08,NGPB09,LRWZHL12,ZL10,K1,K2}. These studies unveil many surprising results, the most 
dramatic of which is the violation of the naive expectation (Fourier's law) which states that the thermal conductance $G$ is inverse proportional 
to the size~$L$ of the system, namely, $G\propto 1/L^{\alpha}$ with ${\alpha=1}$. 

Currently it is well established that in low-dimensional disordered systems, in the absence of non-linearity, Fourier's law  is violated. The
underlying physics is related to the theory of Anderson localization of the vibrational modes \cite{D08,D01,LXXZL12,DL08,LD05,RD08,LZH01,
KCRDLS10a,KCRDLS10b}. On the basis of the prevailing theory \cite{D08,D01} it has been claimed that for samples with ``optimal" contacts ${\alpha=1/2}$, 
while in general $\alpha$ might be larger, say ${\alpha=3/2}$ for samples with ``fixed boundary conditions". Recently the ``optimal" value 
${\alpha=1/2}$ has been challenged by the numerical study of \cite{BZFK13}. These authors found a super-optimal value ${\alpha \sim 1/4}$ for 
moderate system sizes $L$, while asymptotically, in the presence of a pinning potential, $G$ decays exponentially as ${\exp(-\gamma L)}$. 


%%%%%%%

It is obvious that if the final goal 
is to achieve the control of heat flow on the nanoscale, 
first we have to understand the fundamental mechanisms of heat conduction, 
and provide an adequate description of its scaling with the system size for any~$L$, 
including the experimentally relevant cases of intermediate lengths.

%%%%%%%%%%%%%%%%%%%%%%%%%%%%
\sect{The model}
%
We consider a one-dimensional network of $L$ harmonic oscillators of equal masses. 
The system is described by the Hamiltonian 
% 
\begin{equation}
{\cal H} = {1\over 2} P^T P + {1\over 2} Q^T \bm{W} Q 
\label{Hmatrix} 
\end{equation}
%
where $Q^T\equiv(q_1,q_2,\cdots,q_N)$, and $P^T\equiv(p_1,p_2,\cdots,p_N)$ 
are the displacement coordinates and the conjugate momenta. 
The real symmetric matrix  $\bm{W}$ is determined by the spring constants.
Its off-diagonal elements $W_{nm}{=}-w_{nm}$ originate from 
the coupling potential $(1/2)\sum_{m,n}w_{nm}(q_n-q_m)^2$, 
while its diagonal elements contain an additional 
optional term that originate from a pinning potential ${(1/2)\sum_n v_n q_n^2}$ 
that couples the masses to the substrate. 
Accordingly ${W_{nn}=v_n+\sum_m w_{nm}}$. 
For a chain with near-neighbor transitions we use 
the simplified notation ${w_{n{+}1,n} \equiv w_n}$.  


In general the interest is in quasi one-dimensional networks, 
for which $\bm{W}$ is a banded matrix with ${1{+}2b}$ diagonals. 
The heat conduction of such networks has been investigated numerically 
in \cite{BZFK13}, with puzzling findings that have not been explained 
theoretically. From previous work we have seen that the essential physics can be 
reduced to single channel ($b=1$) analysis. On top we would like  
to consider not only weakly disordered network, but also 
the implications of ``glassy" disorder as defined below.   


%%%%%%%%%%%%%%%%%%%%%%%%%%%%%%%%%%%%%%%%%%%%%%%%%%%%%%%%%%
\sect{The disorder}
%
Both the $w_{nm}$ and the $v_n$ are assumed to 
be random variables. The diagonal-disorder due to the pinning
potential is formally like that of the standard Anderson model 
with some variance ${\mbox{Var}(v)=\sigma_{\parallel}^2}$.    
The off-diagonal disorder of the couplings might be 
weak with some variance ${\mbox{Var}(w) \equiv \sigma_{\perp}^2}$, 
but more generally it can reflect the glassiness of the network. 
%
By ``glassy disorder" we mean that the coupling~$w$ has an 
exponential sensitivity to physical parameters.  
For ``random barrier" statistics ${ w \propto \eexp{-B} }$, 
where $B$ is uniformly distributed within $[0,\sigma]$. 
For ``random distances" statistics ${ w \propto \eexp{-R} }$,
where $R$ is implied by Poisson statistics. 
The probability distribution in the latter case is 
%
\beq
P(w) \ \ = \ \ \frac{s}{w_c^{s}} w^{s-1} \ (w<w_c)
\eeq
%
where~$s$ is the normalized density of the sites.
Large $s$ is like regular weak disorder, while small $s$ 
implies glassy disorder that features log-wide distribution  
(couplings distributed over several orders of magnitude). 
The case $s=0$ with an added lower cutoff 
formally corresponds to ``random barriers". 

%%%%%%%%%%%%%%%%%%%%%%%%%%%%%%%%%%%%%%%%%%%%%%%%%%%%%%%%%%
\sect{Localization}
%
The disorder significantly affects the eigenmodes: rather than being extended 
as assumed by Debye, they become exponentially localized. 
We use the standard notation $\gamma(\omega)$ for the inverse localization length: 
%
\beq
\label{locdef}
\gamma(\omega) \ \ = \ \ -\lim_{L\rightarrow \infty}{1\over 2}{\langle \ln(g) \rangle_{\omega} \over L}
\eeq
%
where $\langle\cdots\rangle$ indicates an averaging over disorder realizations, 
and~$g$ is the transmission. The notion of transmission is physically appealing here,  
because we can regard~$\bm{W}$ as the Hamiltonian of an electron in a tight binding model. 
The transmission can be calculated from the transfer matrix $\bm{T}$ of the sample:
%
\beq
\label{gTMM}
g \ \ = \ \ \frac{4|\sin(k)|^2}{|T_{21}-T_{12}+T_{22}\exp(ik)-T_{11}\exp(-ik)|^2}
\eeq
%
where for a single channel ($b=1$) network
%
\be{5}
\bm{T} \ \ = \ \ \prod_{n=1}^{n=L} 
\left(\amatrix{
\frac{\lambda-(v_n + w_{n}+w_{n+1})}{w_{n+1}} & -\frac{w_{n}}{w_{n+1}} \cr 1 & 0 
}\right)
\eeq
%
Above it is assumed that the sample is attached to two non-disordered leads.
Optimal coupling requires the hopping-rates there to be all equal 
to the ``conductivity"~$w_0$, meaning same speed of sound. 



%%%%%%%%%%%%%%%%%%%
\section{Work plan}

{ \bf One parameter scaling for conservative model.-- } We would like to investigate whether one parameter scaling is present in conservative model. In Anderson model all states are localized as opposed to the conservative model which have a single extended state ($k=0$). We would like to know if states near the extended state scaled the same as localized one.

{ \bf Inverse localization length ($\gamma$).-- } We would like to have a theoretical estimate for $\gamma$ both for weak, and ``glassy'' disorder. these estimate will later be used for heat conductance calculation.

{ \bf Heat conductance for intermediate system size.-- } We would like to have a theoretical explanation for the super-optimal value $\alpha \sim 1/4$ numerically presented in \cite{BZFK13}.


%%%%%%%%%%%%%%%%%%%%%%%
\section{Preliminary results} \label{sec:prelim}
%%%%%%%%%%%%%%%%%%%%%%%%%%%%%%%%%%%%%%%%%%%%%%%%%%%%%%%%%%%%%%%

\begin{thebibliography}{99}

\bibitem{LLP03} 
S. Lepri, R. Livi, \& A. Politi, 
%{\em Thermal conduction in classical low-dimensional lattices},
Phys. Rep. {\bf 377}, 1 (2003).

\bibitem{D08} 
A. Dhar, 
%{\em Heat transport in low-dimensional systems}, 
Adv. Phys. {\bf 57}, 457 (2008).

\bibitem{LRWZHL12} 
N. Li, J. Ren, L. Wang, G. Zhang, P. H\"anggi, and B. Li, 
%{\em Colloquium: Phononics: Manipulating heat flow with electronic analogs and beyond}, 
Rev. Mod. Phys. \textbf{84}, 1045 (2012).


\bibitem{COGMZ08}
C.W. Chang, D. Okawa, H. Garcia, A. Majumdar,  A. Zettl, 
%{\em Breakdown of Fourier’s Law in Nanotube Thermal Conductors}, 
Phys. Rev. Lett, {\bf 101}, 075903 (2008).

\bibitem{NGPB09} 
D.L. Nika, S. Ghosh, E. P. Pokatilov, and A. A. Balandin, 
% {\em Lattice thermal conductivity of graphene flakes: Comparison with bulk graphite}, 
Appl. Phys. Lett. {\bf 94}, 203103 (2009).


\bibitem{ZL10} 
G. Zhang, B. Li,
% {\em Impacts of Doping On Thermal and Thermoelectric Properties of Nan-Materials"},  
NanoScale {\bf 2}, 1058 (2010).

\bibitem{K1} 
H. Li, T. Kottos, B. Shapiro, 
% Thermal transport in phononic Cayley-tree networks
Phys. Rev. E  91, 042125 (2015).

\bibitem{K2} 
M. Schmidt, T. Kottos, B. Shapiro,
% Random-matrix-theory approach to mesoscopic fluctuations of heat current
Phys. Rev. E  88, 022126 (2013).

\bibitem{D01}
A. Dhar, 
% {\em Heat Conduction in the Disordered Harmonic Chain Revisited}, 
Phys. Rev. Lett. {\bf 86}, 5882 (2001).

\bibitem{LXXZL12} 
S. Liu, X. Xu, R. Xie, G. Zhang, B. Li, 
% Anomalous Heat Conduction and Anomalous Diffusion in Low Dimensional Nanoscale Systems
Euro. Phys. J. B 85, 337 (2012).
% arXiv:1205.3065v2 [cond-mat.stat-mech] (2012).

\bibitem{DL08} 
A. Dhar, J.L. Lebowitz, 
Phys. Rev. Lett. {\bf 100}, 134301 (2008).

\bibitem{LD05} 
L. W. Lee, A. Dhar, 
Phys. Rev. Lett. {\bf 95}, 094302 (2005).

\bibitem{RD08} 
D. Roy, A. Dhar, 
Phys. Rev. E {\bf 78}, 051112 (2008).

\bibitem{LZH01} 
B. Li, H. Zhao, B. Hu, 
Phys. Rev. Lett. {\bf 86}, 63 (2001).

\bibitem{KCRDLS10a}
A. Kundu, A. Chaudhuri, D. Roy, A. Dhar, J.L. Lebowitz, H. Spohn,
Europhys. Lett. {\bf 90}, 40001 (2010).

\bibitem{KCRDLS10b}
A. Chaudhuri, A. Kundu, D. Roy, A. Dhar, J.L. Lebowitz, H. Spohn,
% Heat transport and phonon localization in mass-disordered harmonic crystals
Phys. Rev. B {\bf 81}, 064301 (2010).

\bibitem{BZFK13}
J.D. Bodyfelt, M. C. Zheng, R. Fleischmann, T. Kottos, 
Phys. Rev. E {\bf 87}, 020101(R) (2013)

\bibitem{DG84}
B. Derrida and E. Gardner, 
J. Physique {\bf 45}, 1283 (1984). 

\bibitem{Abrikosov}
A.A. Abrikosov, 
% The paradox with the static conductivity of a one-dimensional metal,
Solid State Communications {\bf 37}, 997 (1981)

\bibitem{Shapiro}
B. Shapiro, 
%Probability distributions in the scaling theory of localization
Phys. Rev. B 34, R4394 (1986)
%http://journals.aps.org/prb/abstract/10.1103/PhysRevB.34.4394


\bibitem{Izrailev}
F.M. Izrailev, A.A. Krokhin, N.M. Makarov,
%Anomalous localization in low-dimensional systems with correlated disorder
Physics Reports 512, 125 (2012).
% http://www.sciencedirect.com/science/article/pii/S0370157311002936

\bibitem{dubi}
Y. Dubi, M. Di Ventra,
%Fourier’s law: Insight from a simple derivation
Phys. Rev. E 79, 042101 (2009).
% http://journals.aps.org/pre/abstract/10.1103/PhysRevE.79.042101

\bibitem[a]{rmrkA}
Given an eigenstate $\psi$ we define $p_n=|\psi_n|^2$.
The normalization is such that $\sum p_n=1$. 
The participation number PN$=[\sum p_n^2]^{-1}$ 
is a measure for the number of sites that are occupied 
by the eigenstate.

\end{thebibliography}

\end{document}

